% Options for packages loaded elsewhere
\PassOptionsToPackage{unicode}{hyperref}
\PassOptionsToPackage{hyphens}{url}
\PassOptionsToPackage{dvipsnames,svgnames,x11names}{xcolor}
%
\documentclass[
]{article}

\usepackage{amsmath,amssymb}
\usepackage{setspace}
\usepackage{iftex}
\ifPDFTeX
  \usepackage[T1]{fontenc}
  \usepackage[utf8]{inputenc}
  \usepackage{textcomp} % provide euro and other symbols
\else % if luatex or xetex
  \usepackage{unicode-math}
  \defaultfontfeatures{Scale=MatchLowercase}
  \defaultfontfeatures[\rmfamily]{Ligatures=TeX,Scale=1}
\fi
\usepackage{lmodern}
\ifPDFTeX\else  
    % xetex/luatex font selection
\fi
% Use upquote if available, for straight quotes in verbatim environments
\IfFileExists{upquote.sty}{\usepackage{upquote}}{}
\IfFileExists{microtype.sty}{% use microtype if available
  \usepackage[]{microtype}
  \UseMicrotypeSet[protrusion]{basicmath} % disable protrusion for tt fonts
}{}
\makeatletter
\@ifundefined{KOMAClassName}{% if non-KOMA class
  \IfFileExists{parskip.sty}{%
    \usepackage{parskip}
  }{% else
    \setlength{\parindent}{0pt}
    \setlength{\parskip}{6pt plus 2pt minus 1pt}}
}{% if KOMA class
  \KOMAoptions{parskip=half}}
\makeatother
\usepackage{xcolor}
\setlength{\emergencystretch}{3em} % prevent overfull lines
\setcounter{secnumdepth}{-\maxdimen} % remove section numbering
% Make \paragraph and \subparagraph free-standing
\makeatletter
\ifx\paragraph\undefined\else
  \let\oldparagraph\paragraph
  \renewcommand{\paragraph}{
    \@ifstar
      \xxxParagraphStar
      \xxxParagraphNoStar
  }
  \newcommand{\xxxParagraphStar}[1]{\oldparagraph*{#1}\mbox{}}
  \newcommand{\xxxParagraphNoStar}[1]{\oldparagraph{#1}\mbox{}}
\fi
\ifx\subparagraph\undefined\else
  \let\oldsubparagraph\subparagraph
  \renewcommand{\subparagraph}{
    \@ifstar
      \xxxSubParagraphStar
      \xxxSubParagraphNoStar
  }
  \newcommand{\xxxSubParagraphStar}[1]{\oldsubparagraph*{#1}\mbox{}}
  \newcommand{\xxxSubParagraphNoStar}[1]{\oldsubparagraph{#1}\mbox{}}
\fi
\makeatother


\providecommand{\tightlist}{%
  \setlength{\itemsep}{0pt}\setlength{\parskip}{0pt}}\usepackage{longtable,booktabs,array}
\usepackage{calc} % for calculating minipage widths
% Correct order of tables after \paragraph or \subparagraph
\usepackage{etoolbox}
\makeatletter
\patchcmd\longtable{\par}{\if@noskipsec\mbox{}\fi\par}{}{}
\makeatother
% Allow footnotes in longtable head/foot
\IfFileExists{footnotehyper.sty}{\usepackage{footnotehyper}}{\usepackage{footnote}}
\makesavenoteenv{longtable}
\usepackage{graphicx}
\makeatletter
\def\maxwidth{\ifdim\Gin@nat@width>\linewidth\linewidth\else\Gin@nat@width\fi}
\def\maxheight{\ifdim\Gin@nat@height>\textheight\textheight\else\Gin@nat@height\fi}
\makeatother
% Scale images if necessary, so that they will not overflow the page
% margins by default, and it is still possible to overwrite the defaults
% using explicit options in \includegraphics[width, height, ...]{}
\setkeys{Gin}{width=\maxwidth,height=\maxheight,keepaspectratio}
% Set default figure placement to htbp
\makeatletter
\def\fps@figure{htbp}
\makeatother
% definitions for citeproc citations
\NewDocumentCommand\citeproctext{}{}
\NewDocumentCommand\citeproc{mm}{%
  \begingroup\def\citeproctext{#2}\cite{#1}\endgroup}
\makeatletter
 % allow citations to break across lines
 \let\@cite@ofmt\@firstofone
 % avoid brackets around text for \cite:
 \def\@biblabel#1{}
 \def\@cite#1#2{{#1\if@tempswa , #2\fi}}
\makeatother
\newlength{\cslhangindent}
\setlength{\cslhangindent}{1.5em}
\newlength{\csllabelwidth}
\setlength{\csllabelwidth}{3em}
\newenvironment{CSLReferences}[2] % #1 hanging-indent, #2 entry-spacing
 {\begin{list}{}{%
  \setlength{\itemindent}{0pt}
  \setlength{\leftmargin}{0pt}
  \setlength{\parsep}{0pt}
  % turn on hanging indent if param 1 is 1
  \ifodd #1
   \setlength{\leftmargin}{\cslhangindent}
   \setlength{\itemindent}{-1\cslhangindent}
  \fi
  % set entry spacing
  \setlength{\itemsep}{#2\baselineskip}}}
 {\end{list}}
\usepackage{calc}
\newcommand{\CSLBlock}[1]{\hfill\break\parbox[t]{\linewidth}{\strut\ignorespaces#1\strut}}
\newcommand{\CSLLeftMargin}[1]{\parbox[t]{\csllabelwidth}{\strut#1\strut}}
\newcommand{\CSLRightInline}[1]{\parbox[t]{\linewidth - \csllabelwidth}{\strut#1\strut}}
\newcommand{\CSLIndent}[1]{\hspace{\cslhangindent}#1}

%% tempate from https://arxiv.org/abs/2007.12702

\usepackage[toc,page,header]{appendix}
\renewcommand\appendixname{Supplemental Information}
\renewcommand\appendixtocname{Supplemental Information}

\usepackage{minitoc}
\usepackage{framed}

%% set colors 
\usepackage{xcolor}
\newcommand\red[1]{{\color{red}#1}}
\newcommand{\blue}[1]{\textcolor{blue}{#1}}
\newcommand{\black}[1]{\textcolor{black}{#1}}
\newcommand{\purple}[1]{\textcolor{purple}{#1}}
\newcommand{\green}[1]{\textcolor{olive}{#1}}
\newcommand{\white}[1]{\textcolor{white}{#1}}
\newcommand{\bblue}[1]{\textbf{\textcolor{blue}{#1}}}
\newcommand{\bgreen}[1]{\textbf{\color{olive}{#1}}}

\usepackage{comment}
\usepackage{amsthm}
\usepackage{enumitem}
\setitemize{noitemsep,topsep=0pt,parsep=0pt,partopsep=0pt}
\usepackage{caption}
\usepackage{xcolor}
\usepackage{tikz}
\usetikzlibrary{tikzmark}
\usepackage{array}
\setcounter{tocdepth}{2}
\usepackage{multirow}
\usepackage{dcolumn}
\usepackage{placeins}
\usepackage{cancel}
\usepackage{pdfpages}
\definecolor{newcolor}{rgb}{0.45,0.51,0.87}
\usepackage[colorlinks = true,
            linkcolor = blue,
            urlcolor  = blue,
            citecolor = olive,
            anchorcolor = black]{hyperref}
\usepackage{subcaption}
\usepackage{setspace}
\usepackage{amsmath,amsfonts,bm,mathtools,physics,graphicx,longtable,booktabs}
\usepackage{lscape}
\DeclareMathOperator*{\argmax}{arg\,max}
\DeclareMathOperator*{\argmin}{arg\,min}
\newcommand{\simind}{\stackrel{\text{ind.}}{\sim}}
\newcommand{\N}{\mathbb{N}}
\newcommand{\Q}{\mathbb{Q}}
\newcommand{\Z}{\mathbb{Z}}
\newcommand{\R}{\mathbb{R}}
\newcommand{\C}{\mathbb{C}}
\newcommand{\cB}{\mathcal{B}}
\newcommand{\cN}{\mathcal{N}}
\newcommand{\bA}{\bm{A}}
\newcommand{\ba}{\bm{a}}
\newcommand{\bD}{\bm{D}}
\newcommand{\bI}{\bm{I}}
\newcommand{\bM}{\bm{M}}
\newcommand{\bQ}{\bm{Q}}
\newcommand{\bR}{\bm{R}}
\newcommand{\bU}{\bm{U}}
\newcommand{\bV}{\bm{V}}
\newcommand{\bX}{\bm{X}} 
\newcommand{\bW}{\bm{W}}
\newcommand{\bS}{\bm{S}}
\newcommand{\bs}{\bm{s}}
\newcommand{\bY}{\bm{Y}}
\newcommand{\bz}{\bm{z}}
\newcommand{\bZ}{\bm{Z}}
\newcommand{\inty}{\infty}
\newcommand{\inft}{\infty}
\newcommand{\bbeta}{\bm\beta}
\newcommand{\bgamma}{\bm\gamma}
\newcommand{\bdelta}{\bm\delta}
\newcommand{\bepsilon}{\bm\epsilon}
\newcommand{\bLambda}{\bm\Lambda}
\newcommand{\btheta}{\bm\theta}
\newcommand{\bnu}{\bm\nu}
\newcommand{\bmu}{\bm\mu}
\newcommand{\bxi}{\bm\xi}
\newcommand{\bzeta}{\bm\zeta}
\newcommand{\bOmega}{\bm\Omega}
\newcommand{\oracle}{\mathrm{oracle}}
\newcommand{\ridge}{\mathrm{penalty}}
\newcommand{\sdf}{\mathrm{subset}}
\newcommand{\wndf}{\mathrm{wn}}
\newcommand{\pndf}{\mathrm{pm}}
\newcommand{\full}{\mathrm{full}}
\newcommand{\deconf}{\mathrm{deconf}}
\newcommand{\naive}{\mathrm{na\ddot{\i}ve}}
\newcommand{\nonlinear}{\mathrm{nonlinear}}
\newcommand{\E}{\mathbb{E}}
\newcommand{\Var}{\mathrm{Var}}
\newcommand{\Cov}{\mathrm{Cov}}
\DeclareMathOperator*{\plim}{\mathit{p}-lim}
\renewcommand{\d}{\, \mathrm{d}}
\newcommand{\I}{I}
\newcommand{\Prob}{P}
\newtheorem{com}{Comment}
\newtheorem{lem}{Lemma}
\newtheorem{prop}{Proposition}
\newtheorem{thm}{Theorem}
\newtheorem{defn}{Definition}
\newtheorem{cor}{Corollary}
\newtheorem{obs}{Observation}
\newtheorem{ass}{Assumption}
\DeclarePairedDelimiter\floor{\lfloor}{\rfloor}
\DeclarePairedDelimiter{\ceil}{\lceil}{\rceil}
\newcommand{\m}[1]{\begin{bmatrix} #1 \end{bmatrix}}
\newcommand{\snorm}[1]{||#1||}
\newcommand{\indep}{\mbox{$\perp\!\!\!\perp$}}
\DeclarePairedDelimiterX{\inp}[2]{\langle}{\rangle}{#1, #2}
\usepackage{lineno}
\usepackage{afterpage}
\usepackage [autostyle, english = american]{csquotes}
\MakeOuterQuote{"}


%% set colors 

\usepackage{xcolor}
\definecolor{Orange}{rgb}{1,0.5,0}
\definecolor{Red}{rgb}{1,0,0}
\definecolor{Green}{rgb}{0,0.65,0.5}
\definecolor{Purple}{rgb}{0.75,0,1}
\definecolor{babypink}{rgb}{0.96, 0.76, 0.76}
\definecolor{azure}{rgb}{0,0.49,1}
\definecolor{periwinkle}{rgb}{0.8, 0.8, 1.0}
\definecolor{Pink}{RGB}{255, 102, 204}
\definecolor{navyblue}{rgb}{0.0, 0.0, 0.5}
\newcommand{\todo}[1]{\textsf{\textbf{\textcolor{Orange}{[[TODO: #1]]}}}}
\newcommand{\verify}[1]{\textsf{\textbf{\textcolor{red}{[[Verify: #1]]}}}}
\newcommand{\chris}[1]{\textsf{\textbf{\textcolor{azure}{[[Chris: #1]]}}}}
\newcommand{\casey}[1]{\textsf{\textbf{\textcolor{Green}{[[Casey: #1]]}}}}

%% "clever" reference package 
\usepackage[nameinlink, capitalize, noabbrev]{cleveref}
\Crefname{appsec}{appendix}{appendices}

%% custom command for adding in figure / table place holders
\newcommand\PlaceInsert[1]{%
  \begin{center}
  \framebox{\cref{#1} about here.}
  \end{center}
\bigskip}


  \def\fillandplacepagenumber{%
 \par\pagestyle{empty}%
 \vbox to 0pt{\vss}\vfill
 \vbox to 0pt{\baselineskip0pt
   \hbox to\linewidth{\hss}%
   \baselineskip\footskip
   \hbox to\linewidth{%
     \hfil\thepage\hfil}\vss}}

% %% bibliography 
% \usepackage{babel}% Recommended
% \usepackage{csquotes}% Recommended
% \usepackage[style=apa]{biblatex}
% \addbibresource{bunmd_paper.bib} % Syntax for version >= 1.2
% \bibliography{<mybibfile>}% ONLY selects .bib file; syntax for version <= 1.1b

\usepackage[margin = 1in]{geometry}
\usepackage{pdflscape} % to make rotate landscape page
\usepackage{tabularx}
\usepackage{enumitem}

\makeatletter
\@ifpackageloaded{caption}{}{\usepackage{caption}}
\AtBeginDocument{%
\ifdefined\contentsname
  \renewcommand*\contentsname{Table of contents}
\else
  \newcommand\contentsname{Table of contents}
\fi
\ifdefined\listfigurename
  \renewcommand*\listfigurename{List of Figures}
\else
  \newcommand\listfigurename{List of Figures}
\fi
\ifdefined\listtablename
  \renewcommand*\listtablename{List of Tables}
\else
  \newcommand\listtablename{List of Tables}
\fi
\ifdefined\figurename
  \renewcommand*\figurename{Figure}
\else
  \newcommand\figurename{Figure}
\fi
\ifdefined\tablename
  \renewcommand*\tablename{Table}
\else
  \newcommand\tablename{Table}
\fi
}
\@ifpackageloaded{float}{}{\usepackage{float}}
\floatstyle{ruled}
\@ifundefined{c@chapter}{\newfloat{codelisting}{h}{lop}}{\newfloat{codelisting}{h}{lop}[chapter]}
\floatname{codelisting}{Listing}
\newcommand*\listoflistings{\listof{codelisting}{List of Listings}}
\makeatother
\makeatletter
\makeatother
\makeatletter
\@ifpackageloaded{caption}{}{\usepackage{caption}}
\@ifpackageloaded{subcaption}{}{\usepackage{subcaption}}
\makeatother

\ifLuaTeX
  \usepackage{selnolig}  % disable illegal ligatures
\fi
\usepackage{bookmark}

\IfFileExists{xurl.sty}{\usepackage{xurl}}{} % add URL line breaks if available
\urlstyle{same} % disable monospaced font for URLs
\hypersetup{
  pdftitle={Sociodemographic Comparison of Caribbean Hispanic Older Adult Immigrants in the U.S. and Origin Countries},
  pdfauthor={William H. Dow; Chris Soria; Henry T. Dow},
  colorlinks=true,
  linkcolor={blue},
  filecolor={Maroon},
  citecolor={Blue},
  urlcolor={Blue},
  pdfcreator={LaTeX via pandoc}}


\title{Sociodemographic Comparison of Caribbean Hispanic Older Adult
Immigrants in the U.S. and Origin Countries}
\author{William H. Dow \and Chris Soria \and Henry T. Dow}
\date{2024-09-30}

\begin{document}
\maketitle
\begin{abstract}
Caribbean and adjacent Latin American countries are key sources of
Hispanic immigrants to the U.S. There has been rapid growth in the older
adult Hispanic populations both among immigrants in the U.S. and in
their home countries of emigration. This paper supports hypothesis
generation for international comparative Hispanic aging studies by
comparing older adult sociodemographic characteristics of U.S.
immigrants versus those in sending countries. The analysis also provides
context for the global family of health and retirement studies in the
region including the ongoing Caribbean American Dementia and Aging Study
(CADAS) which is collecting harmonized data on healthy aging in Puerto
Rico, Dominican Republic, and Cuba. We analyze census microdata from
these countries along with other major Hispanic Caribbean-adjacent
sending countries including Mexico, Colombia, El Salvador, Guatemala,
and Honduras. We compare older adults in these sending countries to
country-specific immigrant samples in the U.S. American Community
Survey, focusing on socioeconomic differences such as education, as well
as marital status and co-residence patterns related to caregiver
availability. We also examine differences by citizenship and immigration
age to further explore immigrant selectivity patterns. The highly varied
experiences of these cohorts will help inform future comparative
research on Hispanic healthy aging.
\end{abstract}


\setstretch{1.2}
\subsection{Introduction and Background}\label{sec-intro}

Caribbean and adjacent Central and South American countries are key
sources of Hispanic immigrants to the United Sates (Passel 2024). In
2022, people of Mexican origin made up nearly 60\% of the U.S. Hispanic
population, totaling about 37.4 million. Puerto Ricans were the next
largest group at 5.9 million, with an additional 3.2 million living on
the island. Salvadorans, Cubans, Dominicans, Guatemalans, Colombians,
and Hondurans each have populations exceeding 1 million in the United
States (Noe-Bustamante 2023).

These immigrant populations include a rapidly growing subgroup who are
aged 65 and above, among whom there is wide variation in socioeconomic
and caregiving resources. In this paper we explore sociodemographic
variation of U.S. older adult immigrants by country and cohort of
emigration, and compare these U.S. immigrants to the corresponding
cohorts of older adults in their home countries of emigration.

The paper is designed to support hypothesis generation for international
comparative Hispanic aging studies. This includes providing background
context for the global family of health and retirement studies in the
region such as the ongoing Caribbean American Dementia and Aging Study
(CADAS) which is collecting harmonized data on healthy aging in Puerto
Rico, Dominican Republic, and Cuba(Llibre-Guerra et al. 2021). We
analyze census microdata from these countries along with other major
Hispanic Caribbean-adjacent sending countries including Mexico,
Colombia, El Salvador, Guatemala, and Honduras. We compare older adults
in these sending countries to country-specific immigrant samples in the
U.S. American Community Survey, focusing on socioeconomic differences
such as education, as well as marital status and co-residence patterns
related to caregiver availability. We also examine differences by
citizenship and immigration age to further explore immigrant selectivity
patterns. The highly varied experiences of these cohorts will help
inform future comparative research on Hispanic healthy aging.

\subsection{Data and Methods}\label{data-and-methods}

Census data for this study were obtained from IPUMS International, which
provided harmonized datasets for Colombia (2005), Cuba (2012), the
Dominican Republic (2010), El Salvador (2007), Guatemala (2002),
Honduras (2001), Mexico (2020), Puerto Rico (2010), and the United
States (2020). The study focused on individuals aged 65 to 89, as some
datasets, such as Puerto Rico's, capped ages at 89. To ensure
comparability, we standardized the age distribution of the international
population based on U.S. age groups. We also applied weights provided by
IPUMS to make the samples nationally representative.

\subsection{Preliminary Results}\label{preliminary-results}

\subsubsection{Hispanic older adults in their native
countries}\label{hispanic-older-adults-in-their-native-countries}

Table 1 shows sex-specific sociodemographic characteristics among older
adults aged 65 to 89, comparing across current country of residence in
the Hispanic Caribbean and adjacent countries. Rates of current
marriage/cohabitation are substantially higher and cohabitation rates
also vary significantly. In Dominican countries such as Cuba, the
Dominican Republic, and Puerto Rico, women's rates are relatively low
(68\%-69\%) compared to Mexico (75\%) and Guatemala (81\%), yet still
higher than the overall U.S. rate of 48\%. Interestingly, men are much
less likely to be married or cohabitating than women in these countries;
for example, Colombia shows a 30\% gap, with only 38\% of men married or
cohabitating. This contrasts with the U.S., where men (70\%) are more
likely to be married or cohabitating than women (48\%). Regarding
education, Honduras and Guatemala have the highest percentages of women
with less than primary education (84\% and 81\%, respectively), followed
by El Salvador (76\%). Generally, men are slightly less likely to
achieve higher education levels, and notably, in Honduras, no men aged
65 and above hold a university degree.

\subsubsection{Hispanic older adults as migrants in the
US}\label{hispanic-older-adults-as-migrants-in-the-us}

Table 2 highlights the age and marital status of migrants from these
countries living in the U.S. The youngest groups are men from El
Salvador (71.23), Guatemala (70.78), and Honduras (70.78), who are about
two years younger than their counterparts in their native countries
(73.57, 72.99, and 72.92, respectively). The oldest group consists of
Cuban women (73.35), who are slightly younger than U.S. women overall
(73.75), while Cuban men (74.6) are older than the U.S. average for men
(73.13). Interestingly, the pattern observed in their native
countries---where women are more likely to be married or
cohabitating---is reversed among U.S. migrants, with men being much more
likely to be married or cohabitating across all groups. The largest gap
is seen among Colombian migrants, where 88\% of men are married or
cohabitating compared to only 38\% of women, a 50\% difference.

Many of these migrants have low English fluency rates, with women
generally less likely to speak English than men. The lowest fluency
rates are among migrants from the Dominican Republic, where only 60\% of
female and 70\% of male migrants speak English, and Mexico, with 65\% of
women and 75\% of men speaking English. In contrast, migrants from Cuba
(71\% of women and 79\% of men), Colombia (83\% of women and 88\% of
men), and Guatemala (79\% of women and 87\% of men) have higher English
fluency rates. Women are slightly more likely to become naturalized U.S.
citizens than men. Typically, men emigrate to the U.S. at a younger age,
about two years earlier on average. For example, women from Honduras
arrive at an average age of 36.48, while men arrive at 34.42, often
resulting in longer U.S. residency for men---Honduran men report an
average of 37.16 years compared to women's 36.17 years. However,
Guatemala is an exception; despite migrating at a younger age (32.05),
Guatemalan men have spent fewer years in the U.S. (38.75) compared to
women (39.26).

Men are more likely to have a college degree across all groups:
Guatemalan men (12\%) compared to women (8\%), Honduran men (14\%)
versus women (11\%), and Salvadoran men (10\%) against women (5\%). The
largest gender gap is among Colombian migrants, with 24\% of men holding
a college degree compared to 16\% of women. The least educated groups
are Mexican migrants (6\% of men and 4\% of women with college degrees)
and those from the Dominican Republic (9\% for men and 7\% for women).
These patterns contrast with their native countries, where women are
more likely to have a college degree. Overall, migrant groups tend to
have lower educational attainment than the broader U.S. population, with
many having only primary or less education.

\subsubsection{Race and nativity in US
migrants}\label{race-and-nativity-in-us-migrants}

Table 3 compares racial categories among Hispanic and non-Hispanic
groups, focusing on Black, White, and Other racial identities. Notably,
the average ages across these groups are quite similar. Both Hispanic
migrant and native Blacks are much less likely to be married or
cohabitating, with rates of 28\% and 27\%, respectively. In contrast,
individuals identifying as White, whether Hispanic migrant or native,
are more likely to be married or cohabitating. Black migrants are
slightly more likely to speak English but are less likely to become
naturalized citizens. They also tend to immigrate at an older age and
spend fewer years in the U.S., while Whites and those of other Hispanic
races have similar immigration patterns. White Hispanic migrants have
slightly longer U.S. residency. Interestingly, the college degree gap
between Black and White migrants is nonexistent; both groups are equally
likely to earn a degree. This contrasts with native populations, where
only 17\% of Black men hold a college degree compared to 36\% of White
men. Hispanic migrants identifying as a race other than White or Black
are the least likely to have a college degree and most likely to have
less than primary education, with 33\% of women and 32\% of men in this
category having less than a primary education.

\subsection{Discussion}\label{discussion}

\begin{itemize}
\tightlist
\item
  To look for further patterns beyond census data that might help
  explain things like younger ages, younger ages of arrivial, english
  speaking rates, etc.
\end{itemize}

\newpage

\begin{landscape}
\begin{table}[h!]
\centering
\caption{Summary Statistics by Country and Sex For Hispanics in Their Native Countries} 

\begingroup\small
\begin{tabular}{l|l|llllllll}
  \hline
Gender & Demographics & Colombia & Cuba & Dominican Republic & El Salvador & Guatemala & Honduras & Mexico & Puerto Rico \\ 
  \hline
Women & Age & 72.8 & 73.35 & 73.12 & 73.5 & 73.03 & 72.97 & 73.44 & 73.43 \\ 
   & Married/Cohabiting & 0.68 & 0.68 & 0.69 & 0.73 & 0.81 & 0.74 & 0.75 & 0.68 \\ 
   & Less than Primary & 0.58 & 0.2 & 0.69 & 0.76 & 0.81 & 0.84 & 0.41 & 0.21 \\ 
   & Primary & 0.26 & 0.5 & 0.2 & 0.16 & 0.14 & 0.1 & 0.37 & 0.3 \\ 
   & Secondary & 0.06 & 0.23 & 0.06 & 0.05 & 0.03 & 0.04 & 0.1 & 0.36 \\ 
   & University & 0.05 & 0.08 & 0.05 & 0.03 & 0.03 & 0.01 & 0.12 & 0.13 \\ 
   & Unknown & 0.05 & - & - & - & - & - & - & - \\ 
  Men & Age & 73.03 & 73.7 & 73.49 & 73.57 & 72.99 & 72.92 & 73.52 & 74.04 \\ 
   & Married/Cohabiting & 0.38 & 0.47 & 0.38 & 0.38 & 0.51 & 0.44 & 0.47 & 0.43 \\ 
   & Less than Primary & 0.59 & 0.28 & 0.74 & 0.82 & 0.84 & 0.86 & 0.47 & 0.26 \\ 
   & Primary & 0.29 & 0.5 & 0.17 & 0.12 & 0.11 & 0.09 & 0.37 & 0.27 \\ 
   & Secondary & 0.07 & 0.16 & 0.06 & 0.05 & 0.04 & 0.04 & 0.1 & 0.36 \\ 
   & University & 0.02 & 0.05 & 0.03 & 0.01 & 0.01 & - & 0.06 & 0.11 \\ 
   & Unknown & 0.04 & - & - & - & - & - & - & - \\ 
   \hline
\end{tabular}
\endgroup
\end{table}


\begin{table}[ht]
\centering
\caption{Summary Statistics by Country and Sex} 
\label{table2}
\begingroup\small
\begin{tabular}{l|l|lllllllll}
  \hline
Gender & Demographics & cuba & dominican republic & puerto rico & el salvador & guatemala & honduras & mexico & colombia & united states \\ 
  \hline
Women & Age & 75.35 & 72.96 & 73.97 & 72.63 & 72.36 & 72.61 & 73.07 & 73.2 & 73.75 \\ 
   & Married/Cohabiting & 0.34 & 0.31 & 0.33 & 0.34 & 0.38 & 0.35 & 0.45 & 0.38 & 0.48 \\ 
   & English Speakers & 0.71 & 0.6 & 0.89 & 0.68 & 0.79 & 0.75 & 0.65 & 0.83 & 1 \\ 
   & Citizen & 0.84 & 0.7 & - & 0.66 & 0.69 & 0.67 & 0.57 & 0.77 & - \\ 
   & Age at Immigration & 35.63 & 37.38 & - & 36 & 33.13 & 36.48 & 31.8 & 35.38 & - \\ 
   & Years in US & 39.76 & 35.53 & - & 36.63 & 39.26 & 36.17 & 41.27 & 37.83 & - \\ 
   & Less than Primary Completed & 0.11 & 0.31 & 0.15 & 0.38 & 0.3 & 0.23 & 0.4 & 0.13 & 0.01 \\ 
   & Primary Completed & 0.21 & 0.29 & 0.25 & 0.27 & 0.23 & 0.21 & 0.3 & 0.12 & 0.08 \\ 
   & Secondary Completed & 0.5 & 0.33 & 0.48 & 0.31 & 0.38 & 0.45 & 0.27 & 0.59 & 0.66 \\ 
   & University Completed & 0.19 & 0.07 & 0.12 & 0.05 & 0.08 & 0.11 & 0.04 & 0.16 & 0.26 \\ 
  Men & Age & 74.6 & 72.24 & 73.36 & 71.23 & 70.78 & 71.5 & 72.32 & 72.98 & 73.13 \\ 
   & Married/Cohabiting & 0.63 & 0.66 & 0.61 & 0.67 & 0.65 & 0.69 & 0.73 & 0.72 & 0.7 \\ 
   & English Speakers & 0.78 & 0.7 & 0.94 & 0.78 & 0.87 & 0.83 & 0.74 & 0.88 & 1 \\ 
   & Citizen & 0.79 & 0.67 & - & 0.65 & 0.66 & 0.6 & 0.55 & 0.75 & - \\ 
   & Age at Immigration & 34.42 & 36.35 & - & 34.07 & 32.05 & 34.42 & 29.1 & 34.47 & - \\ 
   & Years in US & 40.19 & 35.89 & - & 37.2 & 38.75 & 37.16 & 43.23 & 38.55 & - \\ 
   & Less than Primary Completed & 0.11 & 0.28 & 0.15 & 0.29 & 0.25 & 0.2 & 0.39 & 0.12 & 0.01 \\ 
   & Primary Completed & 0.2 & 0.28 & 0.26 & 0.28 & 0.25 & 0.24 & 0.29 & 0.09 & 0.07 \\ 
   & Secondary Completed & 0.48 & 0.35 & 0.48 & 0.33 & 0.37 & 0.42 & 0.26 & 0.55 & 0.57 \\ 
   & University Completed & 0.22 & 0.09 & 0.11 & 0.1 & 0.12 & 0.14 & 0.06 & 0.24 & 0.34 \\ 
   \hline
\end{tabular}
\endgroup
\end{table}


\begin{table}[ht]
\centering
\caption{Summary Statistics by Country and Sex} 
\label{table3}
\begingroup\small
\begin{tabular}{l|l|p{1.5cm}p{1.5cm}p{1.5cm}p{1.5cm}p{1.5cm}p{1.5cm}p{1.5cm}}
  \hline
Gender & Demographics & Hispanic Black Foreign & Hispanic White Foreign & Hispanic Other Foreign & Non-Hispanic Black Native & Non-Hispanic White Native & Non-Hispanic Other Native & All Native Hispanic \\ 
  \hline
Women & Age & 73.74 & 73.75 & 73.14 & 73.19 & 73.87 & 72.96 & 73.17 \\ 
   & Married/Cohabiting & 0.28 & 0.4 & 0.38 & 0.27 & 0.51 & 0.42 & 0.41 \\ 
   & English Speakers & 0.75 & 0.72 & 0.7 & - & - & 0.99 & 0.99 \\ 
   & Citizen & 0.51 & 0.55 & 0.53 & - & - & - & - \\ 
   & Age at Immigration & 37.29 & 33.49 & 33.79 & - & - & - & - \\ 
   & Years in US & 36.58 & 40.17 & 39.23 & - & - & - & - \\ 
   & Less than Primary Completed & 0.24 & 0.26 & 0.33 & 0.03 & 0.01 & 0.03 & 0.08 \\ 
   & Primary Completed & 0.25 & 0.26 & 0.27 & 0.15 & 0.06 & 0.09 & 0.17 \\ 
   & Secondary Completed & 0.41 & 0.38 & 0.34 & 0.64 & 0.66 & 0.6 & 0.62 \\ 
   & University Completed & 0.1 & 0.1 & 0.07 & 0.18 & 0.27 & 0.27 & 0.13 \\ 
  Men & Age & 72.85 & 73.01 & 72.36 & 72.37 & 73.25 & 72.59 & 72.5 \\ 
   & Married/Cohabiting & 0.55 & 0.69 & 0.68 & 0.52 & 0.72 & 0.63 & 0.61 \\ 
   & English Speakers & 0.8 & 0.79 & 0.78 & - & - & - & 0.99 \\ 
   & Citizen & 0.49 & 0.53 & 0.51 & - & - & - & - \\ 
   & Age at Immigration & 34.78 & 31.13 & 31.42 & - & - & - & - \\ 
   & Years in US & 37.95 & 41.8 & 40.79 & - & - & - & - \\ 
   & Less than Primary Completed & 0.22 & 0.26 & 0.32 & 0.04 & 0.01 & 0.03 & 0.07 \\ 
   & Primary Completed & 0.26 & 0.25 & 0.27 & 0.16 & 0.06 & 0.08 & 0.15 \\ 
   & Secondary Completed & 0.4 & 0.37 & 0.33 & 0.63 & 0.57 & 0.57 & 0.6 \\ 
   & University Completed & 0.11 & 0.12 & 0.08 & 0.17 & 0.36 & 0.32 & 0.18 \\ 
   \hline
\end{tabular}
\endgroup
\end{table}


\end{landscape}

\newpage

\section*{References}\label{references}
\addcontentsline{toc}{section}{References}

\phantomsection\label{refs}
\begin{CSLReferences}{1}{0}
\bibitem[\citeproctext]{ref-llibre-guerra_caribbean-american_2021}
Llibre-Guerra, Jorge J, Jing Li, Amal Harrati, Ivonne Jiménez-Velazquez,
Daisy M Acosta, Juan J Llibre-Rodriguez, Mao-Mei Liu, and William H Dow.
2021. {``The {Caribbean}-{American} {Dementia} and {Aging} {Study}
({CADAS}): {A} Multinational Initiative to Address Dementia in
{Caribbean} Populations.''} \emph{Alzheimer's \& Dementia} 17 (S7):
e053789. \url{https://doi.org/10.1002/alz.053789}.

\bibitem[\citeproctext]{ref-noe-bustamante_key_2023}
Noe-Bustamante, Mohamad Moslimani and Luis, Jeffrey S. Passel. 2023.
{``Key Facts about {U}.{S}. {Latinos} for {National} {Hispanic}
{Heritage} {Month}.''} \emph{Pew Research Center}.
\url{https://www.pewresearch.org/short-reads/2023/09/22/key-facts-about-us-latinos-for-national-hispanic-heritage-month/}.

\bibitem[\citeproctext]{ref-passel_what_2024}
Passel, Mohamad Moslimani and Jeffrey S. 2024. {``What the Data Says
about Immigrants in the {U}.{S}.''} \emph{Pew Research Center}.
\url{https://www.pewresearch.org/short-reads/2024/09/27/key-findings-about-us-immigrants/}.

\end{CSLReferences}




\end{document}
